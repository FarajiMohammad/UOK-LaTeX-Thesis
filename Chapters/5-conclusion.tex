
\titleformat{\chapter}[display]
{\normalfont\huge\bfseries\centering}
{\vspace*{6cm}{‎فصل پنجم}}{5pt}{\Huge}
\chapter{\rl{نتیجه گیری و کارهای آینده}}\label{sec5}
\thispagestyle{empty}
\newpage
\section{نتیجه‌گیری}
در این پایان‌نامه، ما یک روش جدید انتخاب ویژگی چندبرچسبه را با درنظر گرفتن همبستگی برچسب سراسری و محلی پیشنهاد کردیم. همبستگی برچسب سراسری به بهره‌برداری از ساختار زیربنایی فضای برچسب کمک می‌کند. این به مدل اجازه می‌دهد تا روابط و همبستگی‌های بین برچسب‌ها را یاد بگیرید، از سوی دیگر، همبستگی محلی برچسب به ارتباط بین برچسب‌ها در یک پارتیشن محلی خاص اشاره دارد. با درنظر گرفتن همبستگی بین برچسب‌ها و گنجاندن آن‌ها در بازنمایی ویژگی، مدل ویژگی‌هایی را شناسایی می‌کند که بیشترین ارتباط را با برچسب‌ها دارند. علاوه‌بر‌این، ‎\lr{MLFS-GLOCAL}‎ فضای مشترک بین ماتریس ویژگی و ماتریس برچسب را می‌آموزد تا همبستگی برچسب ضمنی را استخراج کند و انتخاب ویژگی را راهنمایی کند. این ماتریس کم‌بُعد از طریق منظم‌ساز چندگانه محدود می‌شود، به این معنی که ساختار محلی مشابهی با داده‌های اصلی دارد و  اطلاعات ارزشمند در داده‌ها حفظ می‌شود. یک الگوریتم تکراری مبتنی بر بهینه‌سازی متناوب برای حل تابع هدف با منظم‌ساز نُرم $\ell_{2,1} $ ایجاد شده است. در نهایت، آزمایش‌های گسترده بر روی مجموعه‌داده‌های چندبرچسبه انجام شده است تا اثربخشی ‎مدل‎ پیشنهادی را در برابر تعدادی از روش‌های انتخاب ویژگی پیشرفته نشان ‌دهد.
\section{پیشنهادهایی برای تحقیقات آتی}
روش‌های پیشنهادی آتی این پایانامه می‌تواند در زمینه‌‌های زیر باشد:
\begin{itemize} 
	
	\item 
	با توجه به اهمیت اطلاعات برچسب در دنیای واقعی می‌توان بر روی داده‌های نیمه‌نظارتی\LTRfootnote{Semi-supervised} \cite{CHAVOSHINEJAD2023109282,8324084} یا گم‌شده \cite{SEYEDI2023562} براساس همبستگی بین برچسب‌ها تمرکز کرد. 
	\item   می‌توان این روش پیشنهادی را بر روی داده‌های چندنمایه\LTRfootnote{Multi-view} چندبرچسبه توسعه داد که در آن داده‌ها با مجموعه ویژگی‌ها یا نماهای متعدد نشان داده می‌شوند و هر نمونه با چندین برچسب مرتبط می‌باشد.
\end{itemize}

%منابع و مأخذ