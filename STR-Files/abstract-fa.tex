\thispagestyle{empty}
\BZarboldE\section*{چکیده}
\BNazTTN
در حوزه‌های مختلف کاربردی، داده‌های چندبرچسبه با ابعاد بالا بسیار موردتوجه قرار گرفته‌اند که دو چالش مهم شامل: نمونه‌ها با ویژگی‌های بُعد بالا و همچنین تعداد زیادی برچسب را به همراه دارد. در انتخاب ویژگی چندبرچسبه، هدف انتخاب زیرمجموعه‌ای از ویژگی‌ها از یک مجموعه است که برای پیش‌بینی چندین برچسب یا دسته مرتبط با هر نمونه بسیار مناسب می‌باشد. با این‌حال، اغلب روش‌های موجود طبقه‌بندی چندبرچسبه، مواردی چون وابستگی‌های برچسب و توزیع نامتعادل برچسب را نادیده گرفته‌اند، هرچند که دارای بینش‌های ارزشمند برای طراحی الگوریتم‌های مؤثر انتخاب ویژگی چندبرچسبه باشند.
در این پایان‌نامه، یک مدل انتخاب ویژگی پیشنهاد می‌شود که با بهره‌گیری از همبستگی‌های سراسری و محلی برچسب‌ها، ویژگی‌های متمایزکننده در سراسر برچسب‌ها را انتخاب ‌کند. علاوه‌بر‌این، با بازنمایی ماتریس ویژگی و ماتریس برچسب در یک فضای مشترک نهان، همبستگی‌های موجود بین ویژگی‌ها و برچسب‌ها را بدست می‌آورد. در این فضای مشترک، الگوها یا ارتباطات مشترکی که در سراسر چندین ‌برچسب و ویژگی وجود دارد، مشخص می‌شود. تابع پیشنهادی شامل نُرم $\ell_{2,1} $ و یک الگوریتم مبتنی بر تکرار و بهینه‌سازی متناوب ‌است تا ضرایب خلوت مورد نیاز برای انتخاب ویژگی چندبرچسبه را بدست آورد. 
ارزیابی روش پیشنهادی بر روی شش معیار ارزیابی و دوازده مجموعه‌داده انجام شده است و نتایج نشان‌دهنده این است که مدل ارائه شده مؤثر، و عملکرد آن از روش‌های مقایسه‌شده بهتر است. \\
\textbf{کلمات کلیدی :}
انتخاب ویژگی، یادگیری چندبرچسبه، همبستگی برچسب، تجزیه ماتریس نامنفی